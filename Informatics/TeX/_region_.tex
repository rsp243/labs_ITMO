\message{ !name(Evklid.tex)}\documentclass{article}
\usepackage[T1,T2A]{fontenc}
\usepackage[utf8x]{inputenc}
\usepackage[14pt]{extsizes}
\usepackage{geometry}
\usepackage{xcolor}
\usepackage{tikz}
\usepackage{graphicx, caption}
\usepackage{wrapfig}
\usepackage{amssymb}

\geometry{top=4em,right=3em,left=3em,bottom=4em}

\begin{document}

\message{ !name(Evklid.tex) !offset(35) }
\linewidth]{src/E.jpg}
  \end{wrapfigure}\\
  \slshape сли у двух треугольников по две стороны соответсвенно равны друг другу (
  \begin{tikzpicture}
    \draw[blue,very thick] (0,0) -- (1,0);
    \node[above] at (0,0) {{\tiny$A$}};
    \node[above] at (1,0) {{\tiny$B$}};
  \end{tikzpicture} =
  \begin{tikzpicture}
    \draw[blue,very thick] (0,0) -- (1,0);
    \node[above] at (0,0) {{\tiny$E$}};
    \node[above] at (1,0) {{\tiny$F$}};
  \end{tikzpicture} и 
  \begin{tikzpicture}
    \draw[dashed,orange,very thick] (0,0) -- (1,0);
    \node[above] at (0,0) {{\tiny$A$}};
    \node[above] at (1,0) {{\tiny$D$}};
  \end{tikzpicture} =
  \begin{tikzpicture}
    \draw[orange,very thick] (0,0) -- (1,0);
    \node[above] at (0,0) {{\tiny$G$}};
    \node[above] at (1,0) {{\tiny$E$}};
  \end{tikzpicture}), и угол заключенный междуними в одном
  \begin{tikzpicture}
    \draw[dashed,red,ultra thick] (0,0) -- (0.5,0.3);
    \draw[red,ultra thick] (0.35,-0.2) -- (0.5,0.3);
    \draw[blue,ultra thick] (0.75,-0.22) -- (0.5,0.3);
    \draw[red,ultra thick] (0,0) arc (220:265:0.5);
    \draw[dashed,ultra thick] (0.4,-0.2) arc (250:295:0.5);
    \node[above] at (0.5,0.3) {{\tiny$A$}};
    \node[left] at (0,0) {{\tiny$D$}};
    \node[right] at (0.75,-0.22) {{\tiny$B$}};
  \end{tikzpicture}
  больше, чем в другом
  \begin{tikzpicture}
    \draw[red,ultra thick] (0.35,-0.2) -- (0.5,0.3);
    \draw[blue,ultra thick] (0.75,-0.22) -- (0.5,0.3);
    \draw[dashed,ultra thick] (0.4,-0.2) arc (250:295:0.5);
    \node[above] at (0.5,0.3) {{\tiny$E$}};
    \node[below] at (0.75,-0.22) {{\tiny$F$}};
    \node[below] at (0.35,-0.2) {{\tiny$G$}};
  \end{tikzpicture}, то сторона
  \begin{tikzpicture}
    \draw[very thick] (0,0) -- (1,0);
    \node[above] at (0,0) {{\tiny$D$}};
    \node[above] at (1,0) {{\tiny$B$}};
  \end{tikzpicture}противолежащая большему углу больше стороны, противолежащей меньшему
  \begin{tikzpicture}
    \draw[yellow,very thick] (0,0) -- (1,0);
    \node[above] at (0,0) {{\tiny$F$}};
    \node[above] at (1,0) {{\tiny$G$}};
  \end{tikzpicture}.
  \upshape
  \begin{center}
    Сделаем
    \begin{tikzpicture}
      \draw[red,ultra thick] (0.35,-0.2) -- (0.5,0.3);
      \draw[blue,ultra thick] (0.75,-0.22) -- (0.5,0.3);
      \draw[dashed,ultra thick] (0.4,-0.2) arc (250:295:0.5);
      \node[above] at (0.5,0.3) {{\tiny$A$}};
      \node[below] at (0.75,-0.22) {{\tiny$B$}};
      \node[below] at (0.35,-0.2) {{\tiny$C$}};
    \end{tikzpicture}=
    \begin{tikzpicture}
      \draw[red,ultra thick] (0.35,-0.2) -- (0.5,0.3);
      \draw[blue,ultra thick] (0.75,-0.22) -- (0.5,0.3);
      \draw[dashed,ultra thick] (0.4,-0.2) arc (250:295:0.5);
      \node[above] at (0.5,0.3) {{\tiny$E$}};
      \node[below] at (0.75,-0.22) {{\tiny$F$}};
      \node[below] at (0.35,-0.2) {{\tiny$G$}};
    \end{tikzpicture} (пр. I.{\tiny23}),\\
    и
    \begin{tikzpicture}
      \draw[red,very thick] (0,0) -- (1,0);
      \node[above] at (0,0) {{\tiny$C$}};
      \node[above] at (1,0) {{\tiny$A$}};
    \end{tikzpicture} =
    \begin{tikzpicture}
      \draw[orange,very thick] (0,0) -- (1,0);
      \node[above] at (0,0) {{\tiny$G$}};
      \node[above] at (1,0) {{\tiny$E$}};
    \end{tikzpicture} (пр. I.{\tiny3}),\\
    проведем
    \begin{tikzpicture}
      \draw[dashed,blue,very thick] (0,0) -- (1,0);
      \node[above] at (0,0) {{\tiny$C$}};
      \node[above] at (1,0) {{\tiny$D$}};
    \end{tikzpicture} и
    \begin{tikzpicture}
      \draw[dashed,very thick] (0,0) -- (1,0);
      \node[above] at (0,0) {{\tiny$B$}};
      \node[above] at (1,0) {{\tiny$C$}};
    \end{tikzpicture}.\\
    Поскольку
    \begin{tikzpicture}
      \draw[orange,very thick] (0,0) -- (1,0);
      \node[above] at (0,0) {{\tiny$C$}};
      \node[above] at (1,0) {{\tiny$A$}};
    \end{tikzpicture}=
    \begin{tikzpicture}
      \draw[orange,dashed,very thick] (0,0) -- (1,0);
      \node[above] at (0,0) {{\tiny$A$}};
      \node[above] at (1,0) {{\tiny$D$}};
    \end{tikzpicture} (акс. I, гип., постр.)\\
    $\therefore$
    \begin{tikzpicture}
      \draw[dashed,blue,fill=blue!30] (0,0) --  (1,0) arc(0:75:0.7) -- cycle;
      \draw[dashed,red,fill=red!30] (0,0) --  (1,0) arc(0:-85:0.7) -- cycle;
    \end{tikzpicture} = 
    \begin{tikzpicture}
       \draw[dashed,yellow,fill=orange!70] (1,-2) -- (1.3,-1) arc(40:140:0.5) -- cycle;
     \end{tikzpicture} (пр. I.{\tiny 5}), но
     \begin{tikzpicture}
       \draw[dashed,red,fill=red!30] (0,0) --  (1,0) arc(0:-85:0.7) -- cycle;
     \end{tikzpicture} <
     \begin{tikzpicture}
       \draw[dashed,yellow,fill=orange!70] (1,-2) -- (1.3,-1) arc(40:140:0.5) -- cycle;
     \end{tikzpicture},\\
     и $\therefore$
     \begin{tikzpicture}
       \draw[dashed,red,fill=red!30] (0,0) --  (1,0) arc(0:-85:0.7) -- cycle;
     \end{tikzpicture} <
     \begin{tikzpicture}
       \draw[dashed,yellow,fill=orange!70] (1,-2) -- (1.3,-1) arc(40:140:0.5) -- cycle;
       \draw[dashed,black,fill=black!70] (1,-2) -- (1.7,-1.6) arc(40:86:0.7) -- cycle;
     \end{tikzpicture},\\
     $\therefore$
     \begin{tikzpicture}
       \draw[very thick] (0,0) -- (1,0);
       \node[above] at (0,0) {{\tiny$D$}};
       \node[above] at (1,0) {{\tiny$B$}};
     \end{tikzpicture} > 
     \begin{tikzpicture}
       \draw[dashed,very thick] (0,0) -- (1,0);
       \node[above] at (0,0) {{\tiny$B$}};
       \node[above] at (1,0) {{\tiny$C$}};
     \end{tikzpicture} (пр. I.{\tiny 19})\\
     но
     \begin{tikzpicture}
       \draw[dashed,very thick] (0,0) -- (1,0);
       \node[above] at (0,0) {{\tiny$B$}};
       \node[above] at (1,0) {{\tiny$C$}};
     \end{tikzpicture} =
     \begin{tikzpicture}
       \draw[yellow,very thick] (0,0) -- (1,0);
       \node[above] at (0,0) {{\tiny$F$}};
       \node[above] at (1,0) {{\tiny$G$}};
     \end{tikzpicture} (пр. I.{\tiny 4})\\
     $\therefore$
     \begin{tikzpicture}
       \draw[very thick] (0,0) -- (1,0);
       \node[above] at (0,0) {{\tiny$D$}};
       \node[above] at (1,0) {{\tiny$B$}};
     \end{tikzpicture} >
     \begin{tikzpicture}
       \draw[yellow,very thick] (0,0) -- (1,0);
       \node[above] at (0,0) {{\tiny$F$}};
       \node[above] at (1,0) {{\tiny$G$}};
     \end{tikzpicture}
   \end{center}
   \begin{flushright}
     ч. т. д.
   \end{flushright}
\end{minipage}

\message{ !name(Evklid.tex) !offset(32) }

\end{document}